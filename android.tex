%% android.tex
%% by Simon Danner

% The Computer Society requires 12pt.
\documentclass[12pt,journal,compsoc]{IEEEtran}
\usepackage[ngerman]{babel}   
\usepackage[utf8]{inputenc}   % UTF8-Kodierung für Umlaute
\usepackage{hyperref}         % Hyperlinks
\usepackage{listings}
\usepackage{graphicx}

\clubpenalty = 10000           % Keine "Schusterjungen"
\widowpenalty = 10000          % Keine "Hurenkinder"

\selectlanguage{ngerman}








\newcommand{\Monat}{%
\ifcase\month
 Monat 0 \or Januar \or Februar \or März  \or April\or Mai\or Juni\or Juli\or August\or 
 September\or Oktober\or November\or Dezember
\fi}


\newcommand{\paperTitle}{
	Connecting Android powered devices to the external world	
}

\newcommand{\paperSubTitle}{
}

\newcommand{\absatz}{
	\parskip 12pt
}

\newcommand{\paperAuthor}{Simon~Danner}

\newcommand{\todo}[1]{
    \textbf{Todo:}#1
}


% korrigiere Silbentrennung % Kann man auch dazu verwendet die Silbentrennung zu deaktivieren
\hyphenation{op-tical net-works semi-conduc-tor}
% Silbentrennung für ein einzelnes Wort deaktivieren geht mit:
% \mbox{wort}


% Bei einem zwei Spalten Layout versucht Latex auf beide Seiten die gleiche Texthöhe hinzubekommen
% Dies verursacht leider manchmal Leeräume z. B. zwischen der Überschrift und dem Text
% Mit der Option \raggedbottom kann dies unterbunden werdne
% Hat aber den Nachteil das das Dokument so einen unregelmäßigen Seitenfuß hat
% Aber immer noch besser wie die Leerräume
%\raggedbottom

\begin{document}

\title{\paperTitle \\ \paperSubTitle }
\author{\paperAuthor,~\IEEEmembership{ AI7 }}% <-this % stops a space

% The paper headers
% \markboth{Journal of \LaTeX\ Class Files,~Vol.~6, No.~1, January~2007}
% {Shell \MakeLowercase{\textit{et al.}}: Bare Advanced Demo of IEEEtran.cls for Journals}

\IEEEtitleabstractindextext{%
	\begin{abstract}
	TODO Abstract
	
	\end{abstract}

	% Note that keywords are not normally used for peerreview papers.
	\begin{IEEEkeywords}
		Android, Hardware, ADK, ADK 2012, AOA, Bluetooth LE 
	\end{IEEEkeywords}
}

\maketitle

\section{Einleitung}


\IEEEPARstart{A}{}ndroid ist 
TODO Einleitung 

\cite{buch}asfd



% Datum
\hfill{\the\day~\Monat, \the\year  }

\section{Allgemeines}
TODO Nutzen, Entwicklung

\section{Anbindungsmöglichkeiten}


\section{ADB}
\subsection{Geschichte}

TODO wann von wem zur Device kommunikation ?? --> IOIO


\section{AOA / OpenAccessory}

\cite{developaoa}

TODO Aufgabe / Nutzen / Ziel
TODO Geschichte
TODO Versionen kurz

\subsection{ADK}

TODO Protokoll
TODO Beispiele
\subsection{ADK 2012}
\subsubsection{Änderungen}

\subsubsection{Audio}
TODO Audio Use Cases

\section{Bluetooth LE}
TODO Geschichte
TODO Nutzen
TODO Implementierung / Code ??


\subsection{NFC TODO}



\section{Ausblick}
Zukunft TODO
\subsection{AOA}

\subsection{Bluetooth}
\subsection{andere Techniken}

\section{Fazit}






\bibliographystyle{IEEEtran}
\bibliography{android_quotes}

\end{document}

